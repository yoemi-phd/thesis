\documentclass[letterpaper,oneside,final]{book}

\makeatletter
\def\bstctlcite{%
\@ifnextchar[{\@bstctlcite}{\@bstctlcite[@auxout]}}
\def\@bstctlcite[#1]#2{%
	\@bsphack
	\@for\@citeb:=#2\do{%
	\edef\@citeb{\expandafter\@firstofone\@citeb}%
	\if@filesw\immediate%
	\write\csname #1\endcsname{\string\citation{\@citeb}}\fi}%
	\@esphack%
}
\makeatother

%% This command is used to control the language of the document.
\newcommand\Langue{french}            

\usepackage{ifthen}
%\usepackage[utf8]{inputenc}

%\usepackage[cyr]{aeguill}
\usepackage{lmodern}
\usepackage[T1]{fontenc}

%%
\ifthenelse{\equal{\Langue}{english}}{
	\usepackage[french,english]{babel}
}{
	\usepackage[english,french]{babel} 
}


\usepackage{graphicx}
% \usepackage{epstopdf}

\graphicspath{{./images/}{./dia/}{./gnuplot/}}

\usepackage{flafter,placeins}
\usepackage{amsmath,color,soul,longtable,colortbl,setspace,xspace,url,pdflscape,cite}

\usepackage[nolist]{acronym}
\onehalfspacing

\usepackage{fancyhdr}
\fancypagestyle{pagenumber}{\fancyhf{}\fancyhead[R]{\thepage}}
\renewcommand\headrulewidth{0pt}
\makeatletter
\let\ps@plain=\ps@pagenumber
\makeatother
\setlength{\headheight}{15pt}
%%
\usepackage{hyperref}
\usepackage{caption}
\makeatletter
\providecommand*{\toclevel@compteur}{0}
\makeatother

\usepackage{subcaption}
\usepackage{siunitx}
\usepackage{amssymb}
\usepackage[bottom]{footmisc}

%%
%% Définitions spécifiques au format de rédaction de Poly.
%% Here we define the Poly formatting.
\RequirePackage[\Langue]{thesispoly}

\newcommand\monTitre{Titre de mon document / Title }
\newcommand\monPrenom{Youssouf}
\newcommand\monNom{Emine}
\newcommand\monDepartement{Mathématiques et génie industriel}
\newcommand\maDiscipline{Mathématiques}
\newcommand\monDiplome{D}
\newcommand\anneeDepot{2024}
\newcommand\moisDepot{Décembre}
\newcommand\monSexe{M}
\newcommand\PageGarde{N}
\newcommand\AnnexesPresentes{O}
\newcommand\mesMotsClef{Liste,de,mot-clés,séparés,par,des,virgules}

\newcommand\monJury{\PresidentJury{F}{Dubois}{Danielle}\\
\DirecteurRecherche{M}{Vigot}{Sébastien}\\
\CoDirecteurRecherche{F}{Couture}{Marie}\\
\MembreJury{M}{Tremblay}{Jean}\\
\MembreExterneJury{M}{Brown}{Joseph}}



\ifthenelse{\equal{\monDiplome}{M}}{
\newcommand\monSujet{Mémoire de maîtrise}
\newcommand\monDipl{Maîtrise ès sciences appliquées}
}{
\newcommand\monSujet{Thèse de doctorat}
\newcommand\monDipl{Philosophi\ae{} Doctor}
}

%%
%% Informations qui sont stockées dans un fichier PDF.
%% Information that is stored in a PDF file.
\hypersetup{
  pdftitle={\monTitre},
  pdfsubject={\monSujet},
  pdfauthor={\monPrenom{} \monNom},
  pdfkeywords={\mesMotsClef},
  bookmarksnumbered,
  pdfstartview={FitV},
  hidelinks,
  linktoc=all
}

\usepackage[titles]{tocloft}

\renewcommand\cftchapleader{\cftdotfill{\cftsecdotsep}}
\renewcommand\cfttabindent{0pt}
\renewcommand\cfttabnumwidth{7em}
\renewcommand\cfttabpresnum{\tablename\ }
\renewcommand\cftfigindent{0pt} 
\renewcommand\cftfignumwidth{7em}
\renewcommand\cftfigpresnum{\figurename\ }

\ifthenelse{\equal{\Langue}{english}}{
	\renewcommand\cftchapfont{CHAPTER }
    \renewcommand\cftchappagefont{}
}{
	\renewcommand\cftchapfont{CHAPITRE }
    \renewcommand\cftchappagefont{}
}

\begin{document}
\bstctlcite{IEEEexample:BSTcontrol}

\frontmatter
\ifthenelse{\equal{\PageGarde}{O}}{\addtocounter{page}{1}}{}
\thispagestyle{empty}%
\begin{center}%
\vspace*{\stretch{0.1}}
\textbf{POLYTECHNIQUE MONTRÉAL}\\
affiliée à l'Université de Montréal\\
\vspace*{\stretch{1}}
\textbf{\monTitre}\\
\vspace*{\stretch{1}}
\textbf{\MakeUppercase{\monPrenom~\monNom}}\\
Département de~{\monDepartement}\\
\vspace*{\stretch{1}}
\ifthenelse{\equal{\monDiplome}{M}}{Mémoire présenté}{Thèse présentée} en vue de l'obtention du diplôme de~\emph{\monDipl}\\
\maDiscipline\\
\vskip 0.4in
\moisDepot~\anneeDepot
\end{center}%
\vspace*{\stretch{1}}
\copyright~\monPrenom~\monNom, \anneeDepot.
\newpage\thispagestyle{empty}%
\begin{center}%

\vspace*{\stretch{0.1}}
\textbf{POLYTECHNIQUE MONTRÉAL}\\
affiliée à l'Université de Montréal\\
\vspace*{\stretch{2}}
Ce\ifthenelse{\equal{\monDiplome}{M}}{~mémoire intitulé}{tte thèse intitulée} :\\
\vspace*{\stretch{1}}
\textbf{\monTitre}\\
\vspace*{\stretch{1}}
présenté\ifthenelse{\equal{\monDiplome}{M}}{}{e}
par~\textbf{\mbox{\monPrenom~\MakeUppercase{\monNom}}}\\
en vue de l'obtention du diplôme de~\emph{\mbox{\monDipl}}\\
a été dûment accepté\ifthenelse{\equal{\monDiplome}{M}}{}{e} par le jury d'examen constitué de :\end{center}
\vspace*{\stretch{2}}
\monJury
%%
\pagestyle{pagenumber}%
\ifthenelse{\equal{\Langue}{english}}{
	\chapter*{DEDICATION}\thispagestyle{headings}
	\addcontentsline{toc}{compteur}{DEDICATION}
}{
	\chapter*{DÉDICACE}\thispagestyle{headings}
	\addcontentsline{toc}{compteur}{DÉDICACE}
}

\begin{flushright}
  \itshape
  À tous mes amis du labos,\\
  vous me manquerez\ldots
\end{flushright}

\ifthenelse{\equal{\Langue}{english}}{
	\chapter*{ACKNOWLEDGEMENTS}\thispagestyle{headings}
	\addcontentsline{toc}{compteur}{ACKNOWLEDGEMENTS}
}{
	\chapter*{REMERCIEMENTS}\thispagestyle{headings}
	\addcontentsline{toc}{compteur}{REMERCIEMENTS}
}
\chapter*{RÉSUMÉ}\thispagestyle{headings}
\addcontentsline{toc}{compteur}{RÉSUMÉ}

\chapter*{ABSTRACT}\thispagestyle{headings}
\addcontentsline{toc}{compteur}{ABSTRACT}
%
\begin{otherlanguage}{english}

Written in English.

\end{otherlanguage}

{\setlength{\parskip}{0pt}%

\ifthenelse{\equal{\Langue}{english}}{%
\renewcommand\contentsname{TABLE OF CONTENTS}%
}{%
\renewcommand\contentsname{TABLE DES MATIÈRES}%
}
\tableofcontents

\ifthenelse{\equal{\Langue}{english}}{%
\renewcommand\listtablename{LIST OF TABLES}%
}{%
\renewcommand\listtablename{LISTE DES TABLEAUX}%
}
\listoftables

\ifthenelse{\equal{\Langue}{english}}{%
\renewcommand\listfigurename{LIST OF FIGURES}%
}{
\renewcommand\listfigurename{LISTE DES FIGURES}%
}
\listoffigures
}

% Liste des sigles et abbréviations / List of symbols and acronyms
\ifthenelse{\equal{\Langue}{english}}{
\newcommand\abbrevname{LIST OF SYMBOLS AND ACRONYMS}
}{
\newcommand\abbrevname{LISTE DES SIGLES ET ABRÉVIATIONS}
}
\chapter*{\abbrevname}
\addcontentsline{toc}{compteur}{\abbrevname}
\pagestyle{pagenumber}
%
\begin{acronym}
\acro{IETF}{Internet Engineering Task Force}
\acro{OSI}{Open Systems Interconnection}
\end{acronym}
%
\begin{longtable}{lp{5in}}
IETF       & Internet Engineering Task Force\\
OSI        & Open Systems Interconnection\\
\end{longtable}


\ifthenelse{\equal{\AnnexesPresentes}{O}}{\listofappendices}{}
\mainmatter

\Chapter{INTRODUCTION}\label{chap:introduction}

\section{Background}\label{sec:background}

\cite{Brydson1999}.


\Chapter{REVUE DE LITTÉRATURE / LITERATURE REVIEW}\label{chap:literature-review}

\Chapter{PREMIER THÈME / FIRST THEME}\label{chap:first-theme}
\Chapter{SECOND THÈME / SECOND THEME}\label{chap:second-theme}
\Chapter{THIRD THÈME / THIRD THEME}\label{chap:third-theme}
\Chapter{CONCLUSION}\label{chap:conclusion}
Texte / Text.

%%
%%  SYNTHESE DES TRAVAUX / SUMMARY OF WORKS
%%
\section{Synthèse des travaux / Summary of Works}
Texte / Text.

%%
%%  LIMITATIONS
%%
\section{Limitations de la solution proposée / Limitations}\label{sec:limitations}

%%
%%  AMELIORATIONS FUTURES / FUTURE RESEARCH
%%
\section{Améliorations futures / Future Research}
Texte / Text.


\ifthenelse{\equal{\Langue}{english}}{%
\renewcommand\bibname{REFERENCES}
\bibliography{thesis}
\bibliographystyle{IEEEtran}
}{%
\renewcommand\bibname{RÉFÉRENCES}
\bibliography{thesis}
\bibliographystyle{IEEEtran-francais}
}
%
\ifthenelse{\equal{\AnnexesPresentes}{O}}{
\appendix%
\newcommand{\Annexe}[1]{\annexe{#1}\setcounter{figure}{0}%
\setcounter{table}{0}\setcounter{footnote}{0}}%
\ifthenelse{\equal{\Langue}{english}}{
	\addcontentsline{toc}{compteur}{APPENDICES}
}{
	\addcontentsline{toc}{compteur}{ANNEXES}
}
\Annexe{Démo}
Texte de l'annexe A\@. Remarquez que la phrase précédente se termine
par une lettre majuscule suivie d'un point. On indique explicitement
cette situation à \LaTeX{} afin que ce dernier ajuste correctement
l'espacement entre le point final de la phrase et le début de la
phrase suivante.


\begin{landscape}
\Annexe{Encore une annexe / Another Appendix}
Texte de l'annexe B\@ en mode «landscape».
\end{landscape}

\Annexe{Une dernière annexe / The Last Appendix}
Texte de l'annexe C\@.

}{}
\end{document}
